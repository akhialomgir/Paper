% !Mode:: "TeX:UTF-8"
\chapter{绪论}
\label{cha:intro}

本章主要介绍了子宫肉瘤研究的相关研究的研究背景及意义,分析了相关课题的研究方法与现状,最后列举了本文的研究目标内容与结构。

\section{研究背景与意义}

作为一种较为罕见的肿瘤,子宫肉瘤相关死亡率在子宫恶性肿瘤中中的比例在16\%以上\cite{commonView},目前的主要治疗方式是以手术为主,手术后一般会通过化疗(少数情况使用放疗)辅助,以确保最后效果。根据SEER数据分析显示,子宫肉瘤患者的中位年龄在65岁左右,发病原因有很多,根据对文献中的数据显示,子宫肉瘤可能与某些遗传基因的突变有关。而子宫肉瘤的复发情况较为常见。子宫肉瘤的复发是指在手术和治疗完成后术后阶段肿瘤重新生长并再次出现的一系列过程。子宫肉瘤的复发并不局限于原发病灶,也可能出现在周边组织内,包括子宫、输卵管、卵巢等等。按照文献显示,\uppercase\expandafter{\romannumeral1}~\uppercase\expandafter{\romannumeral2}期患者5年生存率为59\%,而\uppercase\expandafter{\romannumeral3}期则为22\%,\uppercase\expandafter{\romannumeral4}期为9\%。\cite{commonView}从数据中可以看出,子宫肉瘤预后结果中,复发的概率较高,预后的效果较差。在预防子宫肉瘤的复发中,一些研究表明了术后辅助治疗和放疗能够达到降低复发率的作用,并提高患者的生存率。然而作为一种较少见的恶性肿瘤,目前子宫肉瘤方面尚未有比较公认的预后处理模型出现。为了处理近年各个医院积累的子宫肉瘤的随访与预后数据,利用数据对患者术后复发状况或生存率进行预测与评估,能够更好地帮助患者了解自身身体状况,也能帮助医生对于危险程度较高的患者进行重点关注,从而能够让医疗资源能够更加有效地被利用。因此,本文旨在建立一套使用患者预后数据训练而成的预后模型,并建立相应的UI界面,让医学人士能够方便地管理并使用该系统对于患者的将来的生存状况进行评估。

\section{子宫肉瘤研究发展现状}

目前,子宫肉瘤的研究尚且不是很多,有很多种ML模型可以在其中子宫肉瘤的检测分析中起到一定作用,比较常用的模型主要分为以下两类:图像模型、数据模型。

\subsection{在子宫肉瘤方面使用的图像模型}

图像模型主要用于子宫肉瘤的诊断与治疗方案的提出。对于使用图像进行的分析中,CT和MRI是主要使用的方法。图像模型主要通过人工智能技术对于图像进行分隔,提取出ROI,并用于分类模型。分类模型是用于分类和预测的模型,目前常用于子宫肉瘤的治疗决策中。其中,Transformer与生成对抗网络(GAN)是目前比较常用的模型。Transformer可以实现多模态的医学图像分类。以往使用的深度神经网络由于是基于卷积架构形成,它在图像像素较为清晰,或内容较为复杂时,难以对于图像中结构的远端依赖性有较为明确的认知,对于复杂情况分类效果并不好。然而在使用了自注意力机制的Transformer后,它对远端结构的编码让它拥有了更强的学习表达能力,从而有了更好的分类效果。而生成对抗网络则可以生成与真实数据相似的模拟图像,通过模拟图像与真实图像的对比,生成模型与判别模型的不断迭代提升。GAN生成的图像可以为有限的图像数据添加标注后的新数据,同时其附带的判别模型也可以用于对于医生训练数据的扩充。

\subsection{在子宫肉瘤方面使用的数据模型}

最常用的分类模型中,支持向量机和随机森林首当其冲,这些模型可以用于预测子宫肉瘤的侵袭能力和转移风险,从而为临床医生提供重要的治疗决策参考。

\section{本文的研究内容及目标}

\subsection{研究内容}

本文旨在设计并使用Vue与Flask作为前后端技术栈构建一个简易的患者病情预测平台,调用Python实现的预后模型。从而可以帮助医生预判患者病情,使得医生倾注医疗资源来为高风险患者进行进一步的病情随访,推动医学诊断的数字化,让医生能在这个更便携的平台上开展一系列工作,同时也为医疗服务的集成提供了一条较为有效的工程实践经验。

\subsection{研究目标}

对于本文的研究内容,我制定了以下几条目标:
\begin{enumerate}
    \item [1)] 从SEER数据库获得数据,验证各类模型的前置条件是否满足,测试各类模型方法,评估各类方法的作用、优缺点与效果。
    \item [2)] 实现一个具有登录、授权功能的前后端系统,添加工单功能,同时引入数据导出与导出功能,在后端中集成Python实现的预后模型方法,并提供在线的训练与预测功能,从而能够满足医生与管理员的共同使用。
\end{enumerate}


\section{本文组织结构}

整篇论文一共分为七章。

第一张介绍了子宫肉瘤相关研究的背景与意义,阐述了当前子宫肉瘤相关研究的内容与方向,并说明了本文的研究目标与具体内容。

第二章主要介绍了本文使用的代码环境与技术,描述了相关的曲线或参数的具体含义,并阐释了为何使用这些技术与指标。

第三章是本文筛选并处理数据集的过程,描述了本文中如何从SEER癌症数据集与医院数据集中分别下载并处理数据,
从而能在下面的章节中分析并使用。

第四章分析了数据集中的内容,利用数据进行了生存分析,使用KM曲线与Cox比例风险模型研究各影响因素对生存的影响,并进而验证数据的有效性。

第五章首先使用无监督聚类模型对患者的表型进行分类,并分析了各表型所具有的不同特征与预后效果的不同。用分类得到的患者表型数据作为监督学习的数据源,使用决策树与随机森林模型以判断患者所处的表型,并与使用三年生存率的传统方法进行了对比,评估了两者的效果。

第六章介绍了根据模型实现的系统主要功能及实现过程中的技术细节。

第七章对全文进行了总结,归纳了本文的创新点与具体内容,并指出了本文使用的模型的局限性与改进方向。

