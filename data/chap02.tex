% !Mode:: "TeX:UTF-8"
\chapter{子宫肉瘤预后模型相关技术综述}
\label{chap:env} 

本文的子宫肉瘤预后模型使用Python搭建模型,力求模型能够具有更好的可移植性,并依据这一点构建了相关系统。
系统主要使用Vue.js作为核心框架,并配合以TypeScript和Sass作为技术补充;服务端考虑到模型算法基于Python,
使用轻量级并高度可定制的Flask实现,并使用Docker容器技术让后端的部署更加稳定、便携。

同时,本文亦使用了一些医学方面的常用模型与分析方法,本章主要对使用到的相关技术、生存分析方法与预后模型进行介绍。

\section{环境}

本文使用WSL2的Arch发行版进行开发,Python版本为3.10.9,Vue.js版本为5.0.8,主要依赖为Element-plus。WSL2是基于Windows开发的Linux子系统,使用子系统可以在性能不受限的情况下使用Windows的大部分计算与存储资源,且传统的虚拟机与双启动系统的开销也不存在。相较于直接使用Windows进行开发,使用WSL进行开发能够使用apt、pacman等包管理器,且Arch还具有详细的官方WIKI与社区支持;而比较使用Linux开发,WSL的Remote连接更加稳定,能够更好地使用WIndows独占的部分软件,例如本文使用的SEERStat数据库官方软件,让开发更加便利。

\section{技术栈}

\subsection{Docker}

Docker是一个现今广泛使用的用于开发,运输和运行容器化应用程序的开放平台。使用Docker能够让我们专注于应用程序,
而不是花费大量时间调试基础架构。Docker相比较于之前的虚拟机服务,Docker不需要占用多余的磁盘IO,能够有效减少计算资源的消耗。同时,Docker自包含程序依赖,这意味着Python项目的使用中也可以像js前端框架中一样,使用类似package.json的Dockerfile记录所需要的依赖,不同点是Dockerfile不直接写明依赖,而是用requirements.txt等文件来进行存储。这些优势意味着Docker无论在开发还是后期维护中都提供着很强的便携性与健壮性。
Docker的核心概念是镜像、容器与仓库。其中镜像不难理解,类似于Linux中的镜像。我们一般使用一个基础镜像,譬如Win10、Linux等,这些基础镜像中包含着能够运行容器的最低限度的底层环境,而我们则基于这个基础镜像编写相关的配置,比如导入依赖等,然后将这些配置逐层地添加到镜像中,使用Union FS技术对其进行分层与合层记录。镜像的层化技术能让我们具体地分出环境的各层结构,并依据其共享来减少重复镜像的拉取,从而最大化资源的利用率。而容器则包含程序运行需要的一切环境,轻量化地提供可共享可复制地一致服务,容器层的一切修改都不会作用于底层环境,而容器销毁时随着其生命周期的结束所有更改也会消失,从而提供了copy on write的安全特性。仓库类似Github中的仓库,可以使用Docker命令拉取。
本文使用Docker搭配代码部署平台后,程序可以在上传到Github仓库之后自动部署程序于服务器上,从而实现方便便携的开发。


\subsection{Vue.js}

Vue.js是一个面向用户的框架化、结构化的JS前端框架。它建立于HTML、CSS、JavaScript之上,提供声明式和基于组件的编程模型。声明式表明Vue可以使用模板语法动态渲染HTML,让我们可以基于JavaScript中的各个状态动态地描述输出的HTML文本;而组件化意味着Vue的各个模块是可以高度可重用的,Vue SFC将HTML、CSS、JavaScript组合在一起并封装在一个文件中。Vue使用渐进式框架,可以根据需求使用其支持的特性,包括:去构建化的HTML增强、可以在任何页面上嵌入的组件、单页应用程序(SPA)、全栈服务器端渲染(SSR)、静态站点生成(SSG)、可以面向多种应用(包括桌面、移动、WebGL与终端)。

\subsubsection{Vite}

Vite与Vue-Cli类似,是一个提供项目脚手架与开发服务器的构建工具。不过区别在于Vite并不是构建在Webpack上的,
而是使用浏览器中的的ES模块,这让Vite项目提供了很低的延迟和很高的速度,在大型项目中,Vite拥有着远超Vue-Cli的构建与启动速度。在日常使用中,随着项目的不断增大,基于Webpack的Vue-Cli构建速度一般在20秒左右,而相同体量的Vite项目往往恒定在1秒以下,这在需要经常修改的前端项目中会提供很大的便利。这是由于Vite不绑定服务端而是使用浏览器的原生支持。

但是这样的设计也会带来一些问题,Vite的开发环境需要基于现代浏览器,也就是说至少要支持ES2015,在版本较老甚至只使用CommonJS的浏览器中兼容不全,可能会带来一定问题,不过在如今的生产环境中这个问题很少见;同时它暂时也不支持Vue2;脚手架功能中相较于Vue-Cli有一定删节;最后就是开发与构建工具不同可能会导致一些程序页面构建后与开发服务器上的内容不一致。

\subsubsection{Element-Plus}

Element-Plus是一套基于Vue实现的常用组件库,它提供了较为相当丰富的开源PC组件,这些组件让它成为最富盛名的前端组件库之一,在各类行业的PC前端系统开发中都有着广泛的应用。使用这些封装好的组件能够一定程度上减少开发者自己对于常用组件的再实现与重封装,从而提升开发的速度与稳定性。

Element-Plus的设计原则一共有四条:

\begin{itemize}
    \item{一致 Consistency}: 这表示不但组件的流程与逻辑与生活中使用的一致,而且所有元素和结构亦保证有一致的风格与逻辑
    \item{反馈 Feedback}: 表示用户操作与页面状态都可以让用户感知到不同,从而让用户对网页状态有清晰的认知
    \item{效率 Efficiency}: 说明组件的操作流程直观清晰,用户可以快速而直接地认知到各个结构的用途而不是花额外的时间回忆
    \item{可控 Controllability}: 所有操作都交由用户自身来决策,而且用户可以对已完成或正在进行的操作撤销或终止
\end{itemize}

Element-Plus受到ES2018以及以上的浏览器支持,而如果要对之前的IE或Chrome版本进行支持,则需要用到Babel或者其他工具进行版本控制。

\subsection{Flask}

Flask是一个可拓展性极强的、同时极为轻量的后端架构,它的实现是先基于底层的HTTP与Web服务器功能的封装,再使用WSGI(Python Web Server Gateway Interface)来建立Web应用。这样的结构让Flask在传递请求前,需要先将HTTP报文转换为WSGI所需的字典、响应头部的结构体,前者包含请求的全部信息,而后者则是将要调用的函数。而请求信息则由一个显示应用对象处理,在中小型的Flask后端中,大部分的接口都可以定义在这一个对象中。使用显示对象的原因是在Python中,隐式对象只能包含一个实例,所以使用显示对象能够让应用程序集中在一个文件中。这样的设计让小型的Flask服务不会过于臃肿,而且使用者可以使用寥寥几十行代码构造服务,整体结构也相当清晰。而且相当一部分人工智能模型都在Python中有较为简单与便携的实现方式,使用Python来实现后端可以更为简单地嵌入这些模型,让前后端使用的语言数量减少,从而让系统更加健壮。

\section{生存分析方法}

生存分析是一系列统计分析方法,用于探讨人在特定情况下的,及生存时间的分布由于时间或其他因素的变化趋势。但是生存分析并不仅仅可以用在医学领域,它还可以在商业等多种环境中使用。比如使用生存分析可以探讨会员、订阅等机制的用户使用情况,并让厂商对于如何留住自身的客户有一定作用。其中生存时间并非单纯表示表示对象的存活时间,在医学数据中,由于数据需要得到用户的允许才能使用,往往数据的获取都是通过随访得到的。这意味着对象可能生存了更长的时间,而我们的数据只能确认对象在一段时间的存活,这种现象被称为数据的右删失。而这需要通过模型的修改或者后期调整来去除。

\subsection{KM曲线}

Kaplan-Meier曲线是一种基于对时间轴上的生存患者进行累计从而用来进行时向统计的良好方法,它可以用来评估患者群体的健康状况和治疗效果。在生存分析中,我们可以将用户群体按照特征的不同,比如年龄段、肿瘤大小等,分为多个群体,通过比较群体间生存率的区别,我们可以对特征对于患者的影响有较好的初步结论。同时,KM曲线可以用来判断数据是否能使用Cox比例风险模型。

\subsection{Cox单/多因素分析}

Cox单因素分析一般用来研究单变量中的各个值的关系,它通过类似多项式模型的方法来分析该自变量对因变量的影响,而不纳入其他变量的影响。它主要用来研究该因素对于死亡风险的影响程度。Cox模型定义了风险函数,用来表示一个实体具有相应自变量值时,因变量发生的概率。而Cox多因素得到的p值用来检验该变量对于因变量的影响是否显著,一般p值在0.05以下则表示影响显著,可以纳入。Cox多因素分析则与单因素相仿,研究多个变量对于结果变量的影响程度。

\section{其他预后模型}

\subsection{决策树}

决策树是一种在医学中常用的分类回归模型,它使用树形结构表示决策过程。树上的每一个节点都表示特定范围的特征值,而树的每一个叶子节点都对应一个特定的类。决策树的建立过程中,模型首先把全体数据分为大小相近的若干个子集,然后不断在这些子集中选择最佳的子集,让这些子集具有尽量多的相似特征,直到回归达到最大深度停止。决策树可以同时在分类问题与回归问题上使用。

\subsection{随机森林}

随机森林结合了决策树与随机梯度下降算法,随机地从数据集中抽样,组合成多颗决策树。具体地来说,模型为每棵决策树划分了它使用的数据样本空间,并使用梯度下降算法让它在样本空间上训练,随后得到一系列决策树组成的森林。由于随机森林分为多颗树训练,在样本数较大的大规模数据集中训练速度快于单棵决策树,同时也可以减小样本的方差。随机森林在NLP、医学诊断学、金融分析学中都有较为广泛的应用。

\subsection{聚类+分类}

先使用聚类方法得到带标签的新数据,分析聚类得到的患者表型对应集群的生存表现,并根据这个无监督学习的标签分类来预测患者生存率是本文根据文献\cite{life12060776}使用的新方法。在这个模型中,聚类方法的引入让模型的分类更有可解释性,同时不同患者的表型数据虽然在生存率方面会有不同,但是相近的特征依然存在,从而让模型对于相近的数据会有更相似的结果,连续性更好。相比单纯使用二分类模型的概率,该模型也可以使用聚类统计得到的概率作为参考,整体上得到的分类结果更加细致。
