\chapter{总结与展望}
\label{cha:summary}

本文使用了Kaplan-Meier曲线和Cox因素分析等生存分析方法对于数据库中的子宫肉瘤相关数据进行了分析,了解了子宫肉瘤患者的特征分布,同时选择得到了对于患者生存影响较大的几个因素,确定了选择的特征因素并从数据库中拉取了该数据。在基于以上得到的数据下,本文使用了两类机器学习方法对于患者的预后特征进行了预测,其中第一类方法对于患者的三年生存情况进行预测,并得到了选择的特征的重要性数据;而第二类方法结合了聚类方法和监督学习方法,目的是得到患者表征所属的危险群体,从而从另一个侧面对于患者的生存率进行评估,这种方法的优势在于更有临床价值和分类更为精准。

在得到上述模型后,本文使用Vue.js技术栈与Flask后端框架,力求使用较低的后端语言复杂度构建一个简洁的子宫肉瘤训练预测一体化平台。在这个平台中,高内聚低耦合的组件系统被使用,同时WSGI配合模板功能被利用在系统的构建过程中。系统的设计整体经历了需求分析、接口设计、具体实现等过程,合理按照流程规划了系统的设计思路,从而快速而精确地完成了开发与测试地过程。

然而,本文的工作尚且存在进一步进行的可能。模型方法方面,本文使用的聚类方法较为基础,没有基于危险函数和多表评分等方法进行贴合子宫肉瘤情况的进一步设计,这也导致了聚类的分类在部分特征上有所不足。同时,其他深度学习模型由于硬件限制也没有在工作中使用,未来也可以引入从而测试新的方法的准确度。系统方面,本文的系统考虑到时间原因,并未对多语言进行支持,同时系统的登录仅仅使用较为容易破解的MD5实现,文件的引入使用的是上传到后端再进行请求的形式,这样的过程会导致零时文件的累计。以上问题都是存在的可以改善之处,后续将通过实践进一步改善本文所做的工作。