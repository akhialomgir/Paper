\newpage

{
    % \cleardoublepage% Move to first page of new chapter
    \let\clearpage\relax% Don't allow page break
    \centering\zihao{1}{\textbf{基于机器学习的子宫肉瘤患者的临床病理及预后分析}}

    \vspace{8mm}

    \chapter*{摘\ 要}
}
\addcontentsline{toc}{chapter}{摘\ 要}

子宫肉瘤是源于子宫体部位的一组独立的高度恶性肿瘤,根据原发部位又分为子宫平滑肌肉瘤、子宫内膜间质肉瘤等。该肿瘤属于较为罕见的肿瘤,通常发病于45岁以上的绝经后妇女,虽然已经有一些相关文献对其进行了科普与介绍,关于其临床病理及预后分析的研究依然较少。本文基于SEER癌症数据集使用KM曲线与COX多因素比例风险回归模型对影响其患者生存率的各个因素进行了分析与总结;并根据其预后相关数据进行了多种聚类算法的使用,分析聚类后得到簇所代表的表型具有的特征,从而分析得到子宫肉瘤患者的不同表型特点。由于聚类算法属于无监督学习,难以运用在新的数据上。所以本文建立了相关预后模型,使用决策树与随机森林对SEER上的聚类得到的表型进行训练,从而得到能区分患者所对应表型的预后模型,从而更好地评估患者的生存情况。最后结合医院提供的数据对上述模型进行了验证,总结全文工作与创新点,并展望后续工作。

\vspace{8mm}
\textbf{关键词}: 机器学习, 比例风险回归模型, 决策树, 随机森林, 层次聚类

\newpage
{
    % \cleardoublepage% Move to first page of new chapter
    \let\clearpage\relax% Don't allow page break
    \centering\zihao{1}{Machine learning-based clinicopathological and prognostic analysis of patients with uterine sarcoma}
    \vspace{8mm}

    \chapter*{ABSTRACT}
}
\addcontentsline{toc}{chapter}{ABSTRACT}

Uterine sarcoma is an independent group of highly malignant tumors originating from the body of the uterus, which are classified into smooth muscle sarcoma and endometrial mesenchymal sarcoma according to the site of origin. It is a rare tumor that usually develops in postmenopausal women over 45 years of age. Although it has been popularized and introduced in the literature, there are still few studies on its clinicopathological and prognostic analysis. In this paper, we analyzed and summarized the factors affecting the survival rate of patients based on the SEER cancer dataset using the KM curve and COX multifactor proportional risk regression model; and used various clustering algorithms to analyze the characteristics of the phenotypes represented by clusters after clustering to analyze the different phenotypic characteristics of patients with uterine sarcoma based on their prognosis-related data. Since clustering algorithms are unsupervised learning, they are difficult to apply on new data. Therefore, in this paper, a relevant prognostic model is developed and the phenotypes obtained from clustering on SEER are trained using decision trees and random forests to obtain a prognostic model that can distinguish the phenotypes corresponding to the patients and thus better assess the survival of the patients. Finally, the above models were validated with the data provided by the hospital, summarizing the full work and innovations and looking forward to the follow-up work.

\vspace{8mm}

\textbf{Keywords}: Machine learning, proportional risk regression models, decision trees, random forests, hierarchical clustering